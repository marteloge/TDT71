\chapter*{4) An Evaluation of a Mobile Game Concept for Lectures}

  \begin{itemize}
    \item Evaluation of "Lecture Quiz"
    \item Used in lectures for higher education to promote strong student participation and variation in how lectures are taught.
    \item Multiplayer quiz game
    \item Useful for testing the knowledge level and rehearsing theory
    \item Evaluated in a architecture class with students answering a questionnaire
    \item Focus was usability and usefulness
    \item The results showed that the Lecture Quiz was easy to use and contributed to increased learning. The Lecture quiz was percived as entertaining and half the students claimed they would attend more lectures if such system were used regulary. 
    \item Video games can be used insted of traditional exercises. This approach should motivate the students to put extra effort into the exercises and gives the teachers an oppertunity to monitor the students while they are doing the exercises. Second, video games can be used within lectures to improve the participation and motivation of students. 
    \item Initial main goal of the Lecture Quiz was to develop a game concept that could be used in any course that would: 
      \begin{itemize}
          \item test, motivate and engage the students
          \item The game should be using existing equipment and infrastructure available in the lecture halls at the university
      \end{itemize} 
    \item The game concept was a variant of the Sony PS2 quiz-game Buzz!
    \item Studies shows that wireless technology used in educational setting can increase mental activity, facilitate interactivity, and promote social interaction. 
    \item Characteristics that makes game fun to learn: 
      \begin{itemize}
          \item should promote the approperiate level of challenge
          \item should use fantasy and abstractions to make it more interesting
          \item trigger the players curiosity
      \end{itemize}
      The game supported curiosity, provide challenge, but not fantasy. The lack of fantasy can be compensated by making a multipplayer game were the social interaction becomes an important motivating factor. The level of challange cannot be be tailored to the individual differences.
    \item Two game modes: score distribution and last man standing. These could be used alone or together.
    \item Evaluation of the Lecture Quiz:
      \begin{itemize}
        \item Measured usability, suitability, and usefullness.
        \item Usability: satisfaction, effectiveness, and efficiency
        \item SUS score of 74.25\%. 
        \item Students find the lecture quiz engaging, students learned more, and they would be likely to attend more lectures if such games were used regularly.
        \item It is hard to measure the learning rate, but there was a significant improvement in round number two.
        \item They did not find the game distracting
      \end{itemize}
    \item Experiences from the teachers perspective
      \begin{itemize}
        \item Easy to integrate in the lecture
        \item The students had to install the clients beforehand
        \item No unnecessary delays during the lecture 
        \item "Who wants to be a millionaire"
      \end{itemize}
    \item Conclusion: The main benefit from using such game in lectures is to provide a fun way for students to review topics and to provide useful feedback to the teacher of how much the students have learned.
  \end{itemize}