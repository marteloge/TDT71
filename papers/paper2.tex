\chapter*{2) Massivley Multiplayer Online Role-Playing Games: The Past, Present, and Future}

  \begin{itemize}
    \item 
  \end{itemize}

  \section*{Introduction}
  {\bf MMORPG:} Massivley Multiplayer Online Role-Playing Games. Is played in a persistent online game world that continues to function even when the player logs out. While a player is logged out, other players can continue to play in the virtual world. A player assumes the role of a fictional character and then controls that characters actions. 

  The key aims of the research is to determine the evolutional aspects of MMORPGs. Evaluation of current and future MMORPGs. 

  Current research in the area of MMORPGs fits into:
  \begin{itemize}
    \item The social interactions between players in MMORPGs. 
    \item The different architectures to build MMORPGs. 
    \item The effects of latency on MMORPGs.
    \item Problems that plague MMORPGs. 
  \end{itemize}

  The aim is to determine what is that players liked in past MMORPGs and what they want to see in future games. If this is analysed it can reduce the risk of developing MMORPG. 

  \section*{MMORPG Precursors}
  MMORPGs have evolved from two definable areas: multi-user dungeons (MUDs), and computer role-playing games (CRPGs), both single and multiplayer; they in turn borrowed concepts from pen-and-paper ``Dungeons \& Dragons'' (D\&D).

    \subsubsection*{Dungeons and Dragons}
    D\&D was built from the epic advetures from the Hobbit and Lord of the rings series. This is the first table-top role-playing game. In the game the only boundary is your imagination as you play a character. When players create their character they must make a number of choices that dictate the tasks they can perform, their strengths and weaknesses. First, the player select their race or species for their character. Secondly, players select one of a number of predifined classes (feks fighter or magic class). The dungeon master is an essential component, and is responsible for setting the scene and creating the story, and assumes responsibility for all monsters and friendly characters (NPC). 

    The D\&D rules have served as a framework for the developers of computer games.  
    
    \subsubsection*{Character Development Models}

    {\bf Class-based system:} The player chooses an initial character class when he or she first creates a character. The class chosen defines stengths and weaknesses. The class-bases system allows developers to create a game that requires players to interact and help each other by balancing strengths and weaknesses. 

    {\bf Skill Points-Based System:} You can implement this in two different ways: skill points are given as the player advances instead of levels of a class. The other way is that the player can assign the skill points to various skills and abilites in order to increase their strengths. A other implementation is to gain skill points in their chosen skills by simply using those skills. To improve healing, the player needs to perform a lot of healing. 
    This provides a more personalized character, but it can become unbalanzed when the player learn the best combination of skills. 

    \subsubsection*{Multiuser Dungeons}
    The origin of multiuser dungeons (MUDs) can be tracked back to 1978, when ``MUD1'' was developed. They created an text-based game that could de played simultaneously by multiple users. 

    \subsubsection*{Single Player Computer RPGS}
    RPGs brought a lot of innovative features, many of which continue to be used in MMORPGs. Example is realtime play and the attempt to immerse the player into the world through the use of cultures, day and night cycles, as well as weather effects. All these innovations are taken for granted in todays MMORPGs, but it is interesting to see when they were developed. 

    \subsubsection*{Multiplayer Computer RPGS}
    Is not calles MMORPG because they only cater to a small number of players. ``Neverwinter Nights'' indicated that online games could be successful. ``Diablo'' introduced randomly generated dungeons and items that extended the life of the game by allowing people to replay the game with a different characters and to experience altered dungeons and items. A new version of the ``Neverwinter Nights'' allowed players to create their own content. 

  \section*{History of MMORPGS}

    \subsubsection*{First Generation MMORPGS}
    The first generation of MMORPGs were the first graphical online-role playing games that allowed many players to simultaniously play in the same universe. These games generally had only text-based MUDs to look as models. The next step was to implement user friendly interface without text commands.

    {\bf The first MMORPG:} ``Meridian 59'' is often credited as the first MMORPG, as it had many of the elements that players associate with MMORPGs, at least at a early stage. They used the term ``persistet world'' to describe their game. Meridian 59 was also the first game to have a monthly charging system (business strategy). Meridian 59 graphics quickly outdated, but showed other developers the promising oppertunities for MMORPGS. 

    {\bf Ultima Online:} The first commercially successful MMORPG. Was based on PvP (Player vs Player). It also provided ``crafting'', which allows the players to build their own weapons, armor, and other objects. 

    {\bf EverQuest:} A blueprint for future MMORPGs. It had good 3D graphics and also supported a massive community. EQ brought MMORPGS to the mainstream. EQs main drive was , and is, combat, exploration, and character developement. EQ focused more on being a cooperative model rather than PvP; it encouraged players to work together rather than against each other. EQ also introduced the idea or ``raids'', which is a very large group formed to overcome extreamly difficult encounters that require cooperation of many player simultaneously. 

    {\bf Custamizable Interfaces:} Asheron's Call was known for the originality of its creatures. Used third-party tools for supporting interface enhancements. This is very important because many MMORPGs gets very full of different views with maps and statistics. 

    \subsubsection*{Second Generation MMORPGS}
    The second generation of MMORPGs are not very innovatime, they copied concepts that was successfully implemented in the first generation MMORPGs. The second-generation have improved on the graphics and interface. 

    {\bf Realm versus Realm Combat:} ``Dark Age of Camelot'' biggest innovation was the realm-vs-realm (RvR) combat, which allowed massive battles of human-controlled players. 

    {\bf Instances:} ``Anarchy Online'' created ``instanced dungeons'', more commonly refered to today as ``instances'', which are special zones that generate a new copy, or instance, for each group that enters the zone. This allows players or groups a private copy of the zone, and ensures that there will be no competition for resources or oppertunities to kill enemies in that zone. Solved the problem with people killing each other for fun. 

    {\bf Multiplatform Support:} ``Final Fantasy XI'' was the first successful console-based MMORPG, and the first to servers with players from both consoles and PC users. ``FF XI'' divided its goals into ``quests'' and ``missions''. Quests are like ordinary jobs where the player performs a task to get a reward. Missions advance the storyline and the progress of the player. Also supported the oppertunity to change jobs allows the player to experience the game differently. 

    {\bf One server for all:} ``Eve Online'', a galaxy game, had a lot of empty graphics and enabled to put all players in one server (galaxy).  

    {\bf Player Economy and Crafting:} ``Star Wars Galaxies'' enabled the most extensive set of emotes, moods, and associated animations, which, due to the way it allowed the players to express themselves. A big feature of the game is that almost every item in-game is created by players. 

    {\bf Highly Customizable Characters:} ``City of Heroes'' is said to have the most extensive character-creation system: it is able to customize everything from facial features to the superhero's outfit. Such customization allows players to be visually distinguishable from one another; which many MMORPG fails at: many players with the same equipment end up looking the same. 

    {\bf Refining the MMORPG:} ``EverQuest II'' (EQ2) one of the biggest new features was that players could become tradesmen by spending all their time crafting items to sell to the game community rather than adventuring and leveling up a character class. 

    {\bf World of Warcraft:} 

  \section*{Results and discussion}

    \subsubsection*{Respondents preferences in MMORGPs}
    The survey aimed to discover players preferences for various aspects of all MMORGPs. 

      \begin{itemize}
        \item {\it ``What are the features in MMORGPs you like the most?''} \\
        Fantasy/medieval
        \item {\it ``What are the features in MMORGPs you like the most?''} \\
        Lots of class/skill options, graphics and effects, large world to explore, player vs. player, combat, sosialization.
        \item {\it ``What are the worst issues that plague MMORPGs?''} \\
        Exploits, cheats and Item Duping. Running out of content, player griefing, real-world services. Running out of game content was ranked as a problem, while lots of content was ranked high as a desire in MMORGPs. This means that players are interested in bigger game world with enough content to keep them happy for years. As MMORGP players are already playing these games for long periods of time, they want to see many more activities and goals within these worlds. High latency was alsos amnong top five. Other problems commented was that they felt the level/monster grind was a concern, as it became repetitive and booring to kill monsters over and over again using the same basic strategy in each battle so their character could gain experience and access to new abilities and spells. 
        \item {\it ``Which style of character do you prefer?''} \\
        The most choosed ``skill-based'' or a combination of ``skill/class-based''. A pure class-based character are very repetive. 
      \end{itemize}


  \section*{Future MMORPGs}
      
    \subsubsection*{Improving Existing MMORPG Features}

      {\it ``If your perfect MMORGP were to be released tomorrow, what three existing features would you like to see improved?''}

    \begin{itemize}
      \item {\bf Player vs Player combat (PvP):} less restrictive, unbalanced between classes becuase is sometimes follows the ``rock-paper-scissor''pattern. The need to motivate large-scale battles. 
      \item {\bf The Level Grind:} repetitative actions and strategies. 
      \item {\bf Storyline, World Lore, and Immersion:} distinguish a quest from a job. Job should provide money, while quests should be story-driven.  
      \item {\bf Graphics and Effects:} characters visual apperence and customization. 
      \item {\bf Content and Updates:} Due to the subscription fee, players expect more from developers, especially faster response times. Bugs need to be fixed faster. 
      \item {\bf Classes and character skills:} 
      \item {\bf Technical Enhancements:}
      \item {\bf Item Crafting and player economy:}
      \item {\bf Combat and skills:}
      \item {\bf Downtime:}
    \end{itemize}

    \subsubsection*{Adding New MMORPG Features}

    \begin{itemize}
        \item {\bf Player impact in the game world}
        \item {\bf Player-created-and-controlled content}
        \item {\bf Technical Enhancements}
        \item {\bf Mini games}
        \item {\bf Item crafting and player economy}
        \item {\bf Player aging and death}
        \item {\bf Dynamic environments}
        \item {\bf Dynamic content and quests}
        \item {\bf Realtime combat and damage}
        \item {\bf Nonplayer characters (NPC) interaction}
        \item {\bf Evolution}
      \end{itemize}
