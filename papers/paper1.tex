\chapter*{1) A Brief History of Computer Games}

  \section*{Game Changes the past fifty years}

  \begin{itemize}
    \item Changes in hardware for playing games
    \item Changes in interaction devices
    \item Changes in the software tools available
    \item Changes in game business
    \item Changes in the demographics of the players
    \item Diversification
    \item Changes in the design of games
  \end{itemize}

  \section*{Game History}

  \subsection*{1950-1959}
    \begin{itemize}
      \item First computer game {\bf OXO}, a version of {\bf tic-tac-toe} (1952)
      \item People consider the first interactive computer game to be {\bf Tennis for two} (1958). Effects of gravity.
      \item The first game developers didnt understand the potetial of games because of the hardware/equpment needed.
    \end{itemize}

  \subsection*{1960-1969}
    \begin{itemize}
      \item {\bf Spacewar} (1961)
      \item Companies started to consider the commercial explotation of computer games.
      \item 1966 Sega released the {\bf arcade game Periscope} 
    \end{itemize}

  \subsection*{1970-1979}
    \begin{itemize}
      \item The golden age for arcade games
      \item The first commercial explotation of computer games came through arcade machines.
      \item The arcade machines costed money, making it commercially feasible to exploit computer games. 
      \item The first arcade computer game {\bf computer Space} appeard in 1971, but was not a commercial success
      \item {\bf Pong} was a big commercial success created by {\bf Atari}(1972). Following commercial successes was {\bf Breakout} and {\bf Space Wars}.
      \item {\bf Space Wars} was the first game using vector graphics
      \item in 1978 color was introduced
      \item {\bf Space Invanders} and {\bf Asteroids}
      \item {\bf Death Race} (1976) led to a lot of controversy which led to its end.
      \item Research on interactive televison resulted in the {\bf Odyssy game console} (1972). Could only move some dots on the screen and needed plastic overlays to the TV to add colered playfields.
      \item {\bf Channel F} (1976) making it possible to play different games on the same system.
      \item The big step was when Atari introduced the {\bf VCS system}. The console did not sell well becuase it was expencive and the games was not impressive. It became a success when they introduces Space Invanders. This shows that it is not the hardware that counts, but the games.
      \item The hardware on the CVS was limited (1 kb of memory for program and data) and program was written in assembly. This made it hard to program interesting games. 
      Since there was limited memory, the playing fields was often symmetric. This saved data.
    \end{itemize}

  \subsection*{1980-1989}
    \begin{itemize}
      \item {\bf Pack-Man} (1980), led to coin shortage in Japan.
      \item Many interesting games was introduced in this period: {\bf Pack-Man} (1980), {\bf Donkey Kong} (1981), {\bf Mario Bros} (1983), {\bf The leged of Zelda} (1986), {\bf Final Fantasy} (1987), {\bf Prince of Persia} (1989).
      \item Because of the success of Atari, others followed, as well as companies appeard creating games for the different game consoles. 
      \item Atari failed when making the game {\bf E.T}. Many game companies went bakrupt or stepped out of the game business. Game production moved to Japan. 
      \item Another reason for {\bf the crash} was the introduction of new game computers. Cheap {\bf PC's} appeard and was suited for games because of memory, graphics and sound capabilities. {\bf Commadore 64}.
      \item Games for computers was {\bf easy to copy} because of the floppy disks or cassette tapes. 
      \item The pc games made it possible to {\bf save game progress}. 
      \item The crash of the console market made it possible for other companies to enter the market, like {\bf Nintendo NES} (1985) and {\bf Sega Master System} (1986).
      \item {\bf Super Mario Bros} on NES was a huge success. The NES was more popular, not because of its hardware, but because of its uniqeness of its games. 
      \item The {\bf D-pad} was introduced and replaced the joystick. 
      \item Nintendo introduced the first handheld gaming system {\bf Game Boy} (1989). Came bundled with {\bf Tetris} that became a huge success. For a long time Nintendo was the prime producer of handhelds. 
    \end{itemize}

  \subsection*{1990-1999}
    \begin{itemize}
      \item New game consoles was introduced {\bf Mega Drive/Genesis} and {\bf Nintendo Super NES}. They had better hardware. Special hardware for drawing sprites. Higher screen resoultions.
      \item Nintendo had {\bf Mario} as their main character, Sega introduced {\bf Sonic the Hedgedog}.
      \item There were other systems around, but Sega and Nintendo had the Majority of the market.
      \item The newcomer was in the game console was {\bf Sony} who relesed {\bf PlayStation} (1994). PlayStation had faster processor, more memory and special hardware for 3D. 
      \item Games for PlayStation were easier to programming. Resulted in higher game production. 
      \item Players wanted more complex games and it resulted in a {\bf change in the game business}. Huge game budgets became common. 
      \item PC became mature. Many great games were produced like {\bf Sim City}.
      \item PC's could stream video and music from CD. This led to a {\bf new generation of games} that relied on good integration of video and sound. 
      \item PC's had the advantages with the mouse and keyboard. 
      \item PC's had {\bf modem}. This led to the rise of the many massive multiplayer online role playing games  {\bf MMORPG}. 
      \item {\bf The nerds takes over!} The games on the PC was hard to install making the most interested nerds the only ones playing. 
      \item PC's had {\bf different hardware specs}, making it hard to program games for all the different hardware specs. 
      \item Windows 95 and the release of {\bf DirectX} (1995) abstracted away underlying hardware. 
      \item {\bf Doom} ``the first'' first-person shooter game with fake 3D. 
      \item Because of the success of the 3D like game Doom, there was an increased interest in 3D graphics cards for PC.
      \item Handhelds got a new generation, {\bf Game Boy Color} (1998). It had better spec than the older one, and it could communicate wither other devices. 
      \item {\bf Pokemon} became a important game for the Game Boy Color. Pokemon added the important collecting aspects. Nintendo introduced different colors, and players had to communicate with other players in order to collect all possible pokemons. 
    \end{itemize}

  \subsection*{2000-2009}
    \begin{itemize}
      \item Sega decided to stop the production of game consoles in 2001, but continued to make games for other game consoles. 
      \item In 2000 the {\bf PlayStation 2} was released. Better specs, excellent sound qualities, network adapter, DVD player.
      \item The DVD feature in in PS2 was responsible for a quite early sales because PS2 was cheaper than other DVD players. The problem was that people that bought PS2 as a DVD player did not buy any games. PS2 had backwards compatible with PS1. PS2 games was hard to program, but a huge amout of popular games was lauched and the PS2 became popular. 
      \item Nintendo followed in 2000 with its {\bf GameCube}. They did not focus on specs and was weaker than its competition, but a huge amount of great games (casual games) and the low prize made it popular. 
      \item In 2001 Microsoft entered the market with {\bf Xbox}. It is rumored that Microsoft started their own console after Sony refused to use DirectX. The Xbox was a powerful machine that basicly was a PC in a console box. One of the top titles was {\bf Halo}. Good in the {\bf online domain}.
      \item Xbox was lauched late, and had a hard competion with the available SP2 with a lot of available games. 
      \item Creating games became for new consoles became a more complicated and expensice. Players wanted better graphics, movies, sound tracks, and increased playing time. {\bf Game budget increased} because they needed bigger teams with programmers, artists, and much more. 
      \item PC's was getting better specs and the hardcore gamers prefered the PC over the game consoles. This araised the problem to fit the different specs on the different computers. 
      \item Another problem with PC games was that they was easily cracked and it became harder to make money on PC games. 
      \item {\bf The sims} (2000)
      \item The rise of MMORPG. {\bf World of Warcraft} (2004).
      \item The rise of casual games. Faster internet connection and free games making money on advertisement. Easy to learn and short game play. {\bf Bejeweled} (2001).
      \item The rise of social networking and games. {\bf Farmville}. 
      \item {\bf Game Boy Advance} (2001) and {\bf Game Boy Advance SP} and {\bf Nintendo DS} with double screen, and {\bf PlayStation Portable (PSP)}, and {\bf PSP Go} with downloadable content (App store approach).
      \item People started playing games on mobile devices. The release of Iphone in 2007 had a huge effect on mobile gaming. Touch screen and accelerometer raised new game types. Revenue to developers, individual small teams could make games. Not dependent on big publishers. 
      \item {\bf Xbox 360} (2005) and {\bf PlayStation 3} (2007)
      \item Xbox with online achievement system {\bf Gamescore and ranking}.
      \item {\bf Nintendo Wii} (2006) took a different direction. Weak machine, but the controllers was reveloutionary buy registering movements and the machine was cheap. They later releaset {\bf Wii fit}. 
    \end{itemize}

  \subsection{2010-2011}
    \begin{itemize}
      \item Microsoft and Sony responded to the success of Wii. Sony introduced {\bf Kinect}
      \item The big companies have big launch costs because they sell the consoles lower than the production cost, making a lot of time for recovering. They are looking at {\bf cloud} for future gaming. The broadband are not capale for this at this point. 
      \item Mobile gaming is getting increased attention. 
      \item The new hype is tablets. 
      \item {\bf Stereostopic 3D}. Modern televisions are now capable of displaying 3D. The use of glasses are not feasible. 
      \item Games are now played by different genders and ages, not just by the nerds. 
    \end{itemize}

  \section*{Example: Tennis}

    \begin{itemize}
      \item Graphics
      \item Opponents AI
      \item Storyline and setting
      \item Internet connectivity
      \item Interface and control
    \end{itemize}

  \section*{Changes in Graphics}
    \begin{itemize}
      \item Probably the most dominant change in games over the past 50 years is the change in graphics. 
      \item Sprites are small bitmaps that is initially the standard way to draw objects on the screen.
      \item Consoles had special hardware to quickly draw sprites. 
      \item Parallax scrolling was a trick in simulating 3D. images in the background moved slower than the images in the foreground. 
      \item Isometric projections. look at the game from a andgle of 45 degrees. 
      \item Graphics looked nice, but they didnt spend a lot of money on AI. When the player sees very realistic graphics he/she expects also very realistic behaviour. 
    \end{itemize}

  \section*{Changes in interaction devices}

  \section{Changes in demographics}

  \section*{Changes in gameplay}
    \begin{itemize}
      \item Adrenaline rush
      \item Highscore list
      \item difficulty level. Gradually get harder.
      \item The internet. 
    \end{itemize}

  \section*{Changes in business}
    \begin{itemize}
      \item Game design is becomming a art, and are getting more complex
      \item Artists, music, pictures, graphics, sound, effects, history, programmers, marketing, etc. 
      \item Growth in games for smartphones and other mobile devices. 
      \item Business models: advertisements, buying benefits with money.  
    \end{itemize}

  % Hardware vs. Game uniqness 





