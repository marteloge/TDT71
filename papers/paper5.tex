\chapter*{What makes things fun to learn? Heuristics for Designing Instructional Computer Games}

  \section*{Introduction}
  \begin{itemize}
    \item What makes computer games fun?
    \item This paper provides a set of heuristics or guidelines for design of instructional computer games.
    \item Focus on what makes educational games fun, not what makes them educational.
    \item Essential Characteristics of good computer games:
        \begin{itemize}
            \item Challenge
            \item Fantasy
            \item Curiosity 
        \end{itemize} 
  \end{itemize}

  \clearpage
  \section*{Characteristic 1: Challenge}
    \begin{itemize}
      \item In order for a computer game to be challengeing, it must provide a goal whose attainment is uncertain. A number of important consequences follow up from this princple:
      \item {\bf Goal:} A single feature of games that correlated most strongly with preference was wether or not the game had a goal.
        \begin{itemize}
          \item {\bf Not all goals are equally good:}
            \begin{enumerate}
              \item uing the skill being taught was a means to achieving the goal, but was not the goal itself.
              \item The goal was part of an intrinsic fantasy
              \item Because of the child hero, the goal was presumably one with which the child reader could identify.
            \end{enumerate}
          \item {\bf Rich, responsive environments may be unappealing unless they provide an approperiate goal.}
            \begin{itemize}
              \item Simple games should provide an obvious goal. Goals can be made obvious or compelling by the use of visual effects or fantasy.
              \item A complex environment without built-in goals should be structured so that users will be able to easily generate goals for appriperiate difficulty. 
              \item The best goals are often practical or fantasy goals (like reaching the moon in a rocket), rather then simply goals of using a skill (like doing arithmetric problems).
              \item The players must be able to tell wether they are getting closer to a goal (performance feedback and informative feedback).
            \end{itemize}
        \end{itemize}
        \item {\bf Uncertain outcome:}
          \begin{itemize}
            \item A game is usually boring if the player is either certain to win or certain to lose. 
            \item There are four ways to make outcome of a game uncertain:
            \begin{enumerate}
              \item {\bf Variable difficulty level:} determined automatically, chosen by players, determined by the opponents skill.
              \item {\bf Multiple level goals:} score-keeping, speed responses.
              \item {\bf Hidden information}
              \item {\bf Randomness}
            \end{enumerate}
          \end{itemize}
        \item {\bf Self-esteem:}
          \begin{itemize}
            \item Success in a computer game, like success in any challenging activity, can make people feel better about themselves. 
            \item The opposite side of this principle is of cource failure in a challenging activity. Too much failure can lower a person self-esteem and can decrease a persons desire to play the game again.
            \item Performance measure feedback should be presented in a way that minimize the possibility of self-esteem damage. 
            \item There needs to be a balance between the success and failure: aslo called an approperiate difficulty level.
            \item 
          \end{itemize}
      \end{itemize}

  \section*{Characteristic 2: Fantasy}
    \begin{itemize}
      \item Games that include fantasy show or evoke images of physical objects or social situations not actually present.
      \item Non-fantasy games involve only abstract symbols
      \item {\bf Intrinsic and extrinsic fantasies:}
        \begin{itemize}
          \item One relatively easy way to try to increase the fun of learning is to take an existing curriculum and overlay it with a game in which the player progresses towards some fantasy goal, or avoids some fantasy catastophe, depending only on whether the players answers are right or wrong. 
          \item In feks hangman, the fantasy depends on the use of the skill, but not vice versa. This is called a {\bf Extrinsic fantasy}. In contrast, with {\bf intrinsic fantasy} were the skill also depends on the fantasy. 
          \item Most {\bf extrinsic fantasies} depend only on whether or not the skill is used correctly.  
          \item In {\bf intrinsic fantasies}, not only does the fantasy depend on the skill, but the skill also depends on the fantasy. This usually means that problems are presented in terms of elements of the fantasy world (Feks Dart and balloons). In intrinsic fantasy, the events in the fantasy would usually depend not just on whether the skill is used correctly, but on how its use is different from the correct use. Feks in Darts, the players can see graphically wheter their answears are too high or too low and if so by how much.
          \item In games were players are able to exploit analogies between their existing knowledge about the fantasy world and the unfamiliar things they are learning. 
          \item One potential problem with extrinsic fantasies involving a {\bf fantasy catastrophy} is that the catastrophy may be so interesting that players try to get wrong answers so they can see it. 
        \end{itemize}
      \item {\bf Emotional aspects of fantasy:}
        \begin{itemize}
          \item It is very difficult to know what emotional needs people hav eand how these needs might be partially met by computer games. It seems fair to say, however, that computer games that embody emotionally-involving fantasies like war, destruction, and competition are likely to be more popular than those with less emotional fantasies. 
          \item one obvious consequence of the importance of emotional aspects of fantasies, is that different people will find different fantasies appealing. Feks games with wars and guns can affect the genders interest in the game.  
        \end{itemize}
    \end{itemize}

  \section*{Characteristic 3: Curiosity}
    \begin{itemize}
      \item Curiosity is the motivation to learn, independent of any goal-seeking or fantasy-fulfillment. 
      \item Computer games can evoke a lerners curiosity by providing environments that have an optimal level of informal complexity. In other words, the environment should neither be too complicated nor too simple with the respect to the learners existing knowledge. 
      \item In general, an optimal complex environment will be one were the learner knows enough to have expectations about what will happen, but were these expectations are sometimes unmet. 
      \item Challenge refers to what a player can do; complexity refers to what a player can understand.
      \item {\bf Sensory curiosity:}
        \begin{itemize}
          \item Sensory curiosity involves the attention attracting value of changes of patterns in the light, sound, or other sensory stimuli of an environment. 
          \item Computer games can appeal to sensory curiosity by the use of audio and visual effects: 
            \begin{enumerate}
              \item As decoration (decorative visual effects)
              \item To enhance fantacy (like circus music to be set in the environment)
              \item As reward (when the player scores)
              \item As a representation system (to represent information more effectively than with words an numbers)
            \end{enumerate}
        \end{itemize}
      \item {\bf Cognitive curiosity:}
        \begin{itemize}
          \item Cognitive curiosity can be thought of as a desire to bring better "form" to ones knowledge structures.
          \item A way of engaging learners curiosity is to present just enough information to make their existing knowledge seem incomplete and inconsistent. The learners are then motivated to more in the order to make their cognitive structures better-formed. Feks when you read a murder mystery beside the last chapter, you have an idea of who the murder is, but you have a cognitive motivation to bring completeness to your knowledge structure by finding out who the murder was. 
          \item For instance, students may know that flowers need sun to grow, but they might not know that some plants can live in the dark. 
        \end{itemize}
        \item {\bf Informative feedback:} One way of making environments interestingly complex is to make them responsive.
          \begin{itemize}
            \item  To engage a learners curiosity, feedback should be surprising. The easy way of doing this is by using randomness. 
            \item To be educational, feedback should be constructive.  
          \end{itemize}
        \item One extreamly powerful tool for tailoring feedback to specific learners and thus maximizing their curiosity is to maintain online cognitive models of the users. 
    \end{itemize}
