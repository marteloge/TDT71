\chapter*{3) Pervasive Games: Bringing Computer Entertainment Back to the Real World}

  \section*{Introduction}
  Computer games have some advantages that make them more popular than traditional games. {\it First}, they attract people by creating the illusion of being immersed in an imaginative virtual world with computer graphics and sound. {\it Second}, the goals of computer games are typically more interactive than that of traditional games, which brings players a stronger desire to win the game. {\it Third}, computer games, usually designed with an optimal level of information complexity, can easily provoke players' curiosity. COnsequently, computer games intrinsically motivate players by bringing them more fantasy, challenge and curiosity, which are the three main elements contributing the fun in games. 

  However the development of computer games has often decreased the users' physical activities and social interactions. Computer games focus the users' attention mainly on the computer screen or 2D/3D virtual environments, and players are bound to using keyboard, mice, and game pads while playing, thereby constraining interaction. {\bf To address this problem}, there is a growing trend in todays games to bring more physical movements and social interaction into games while still utilizing the benefits of computing and graphical systems. Thus, the real world is coming back to computer entertainment with a {\bf new game genre}, referred to as {\bf Pervasive games}, stressing the pervasive and unbiqutious nature of these games: Pervasive games are no longer confined to the virtual domain of the computer, but integrate the physical and social aspects of the real world. 

  \section*{Pervasive Game Genres}
    We will now look at different types of pervasive games setting focus on different aspect og the gaming experience.
    It is given as an example to give the reader an idea of scope and diversity of this exciting and emerging field. 

    \subsection*{1) Smart Toys}

    Due to their shapes or forms, toys might suggest in which ways they should be played with, but in contrast to games, they are not bound by any rules or limitations on their use. 

    Most current realizations include traditional physical toys equipped with simple sensing technology linked to computer logic. The logic reacts to the changes in the toys physical state by either playing sounds or displaying graphical information. 

      \subsubsection*{Zowie Playsets}
      Consists of a physical toy with movable pieces accompanied by a CD-ROM an a serial connection to a PC. The toy has integrated sensors that transmit the state of the playing pieces of the computer application. Offers several modes of play (discovery, exploration play, hands-on active play, and problem solving play).

      \subsubsection*{Story Toy}
      A rather similar approach to the zowie playsets. The StoryToy is a storytelling environment that uses an audio replay engine in conjunction with a tactile user interface based on a sensor network. The tactile interface consists of an animal farm with a multitude of animals as actors. The study demonstrated that audio feedback alone already creates an enjoyable level of interactivity, in addition to the traditional free play that the toys provides. 

      \subsubsection*{SenToy}
      Present an interesting affective control toy, named SenToy, used to control a synthetic character in the computer game, Fantasy A. SenToy is a doll wirelessly connected to a PC. It allows players to influence the emotions of a synthetic character in FantasyA. Via SenToy, by using gestures associated with anger, fear, surprise, sadness, and joy, players influence the emotions of the characters they control in the game. 

      Players in the game FantasyA have to master SenToy and exhibit a particular set of emotions and perform a set of actions in order to evolve in the game.

      Sensors are registering hands in front of eyes, walking, acceleration.

    \subsection*{2) Affective Gaming}
    One of the games of pervasive computing is to create context-aware applications that will adapt their behavior to information collected from the environment. 

    The {\it who} and {\it where} of a players context has been harnessed (utnyttet)  in some location-based pervasive games, while the {\it what} and {\it when} are common elements of most traditional games.
    Capturing {\it how} a player is feeling at any given moment ant integrating this very personal representation of context into a game is {\bf the goal of affective gaming}.

    Affective computing is described as ``computing that relates to, arises from, or deliberately influences emotions'' and affective gaming aims to integrate a players emotional state into the game so that the game environment can adapt to create a magical gaming experience.  

    Sensing an individuals emotional state is a complex and open research problem; but sensing certain aspects of a players experience while engaged in entertainment technologies is more manageable. The most common approach to sensing an affective state is via sensors that measures the users changing in physiological activity (skin sensors). Embedding sensors in chairs or game controllers, so that users can interact very naturally with their entertainment technologies. Other methods is thermal cameras, voice analysis, or facial expression analysis.  

      \subsubsection*{Utilizing Affect in a Game Environment}
      Once a players affective state has been sensed, it can be used to inject personality into a game environment, resulting in an environment that meaningfully responds to a players context rather than to preconceived gaming challenges. 

      {\bf S.M.A.R.T Braingames:} uses real video games played on a PS2 integrated with NASA technology. The system determines whether the user is in the desired brain state by using brain waves measured by EEG, and adjusts accordingly. If the user maintains the desired brain state, he or she gains full control of the game controller. If not, the speed and steering control decrease. Basically, as the player maintains focus, the game responds, and when the player loses focus, ground is lost. The game was used to train patients to achieve a desired brain state, and not as a source of fun itself. The principle used in the game can be used to develop games that dynamically responds to how the player is feeling in order to create a more engaging experience. 

      {\bf Physiology as direct input:} Instead of using a players context to manipulate the game environment, one could use it as a direct and natural input to a game. For example {\bf brainball} is a game where brain waves (from EEG) are used to alter the direction in which the physical ball rolls on a physical table. Players sit across each other and must relax in order to make the ball move towards the opponent. 
      {\bf AffQuake} alters game play in the popular first-person shooter game via the players galvanic skin response, sensed through metal contacts on the hands or feet. When a player gets ``forskrekket'', the players avatar is also ``forskrekket'' and jumps back. It also relates the size of the players avatar to the players arousal (``opphisselse''). In ``Relax To Win'', a player controls the speed of a racing dragon via galvanic skin responses. As a player relaxes, the dragon moves faster. 

    \subsection*{3) Augmented (utvidet) Tabletop Games}
    While affecting games mainly focuses on using physiological state parameters to exchange information with virtual game elements, augmented tabletop games also integrate the states of the players as central to the game experience. Augmented tabletop games do not serve as an input to the virtual game logic alone, but also add the richness of the social situation to the virtual domain. Traditional popular tabletop games are Chess and Go. Their continuing success can clearly be attributed to the direct interaction and communication between the players, who sit together around the same table, facing each other at an intimate distance. 
    
    Many claims that the drawback to traditional computer games is the lack of social interaction in a face-to-face setting, which tabletop games provide. At the same time, computer games offer the attractions of computer technology, which tabletop games lack. Therefore, it is only a natural evolution to combine the benefits of computer and tabletop games into a novel type of augmented tabletop games that sets out to provide new and engaging gaming experiences. 

      \subsubsection*{The STARS platform:} The platform consists of a dedicated hardware setup of devices such as a public vertical displays and personal digital assistants centered on a smart interactive table. Physical playing pieces provide a tangible interface that feels similar to the interface of a traditional board games. The playing pieces are detected by an overhead camera that also determines the positions of the players by tracking their hands as they reach over the tables surface. Additionally, an integrated RF-ID antenna detects physical tokens placed on the table surface. Playing pieces remains the primary interaction device during game-play because they provide the most natural interface to a board game.

      The board changes during game progress. Auto-rotation boards able all players to see the same view of the board. 

      {\bf KnightMage} is deals with the exploration of a dungeon filled with treasures, equipments and monsters. In the game players must cooperate to survive against monsters in the dungeon. The combination of competitive and cooperative play is also found in traditional board games like Risk.

      \subsubsection*{False Prophets}
      Is a hybrid board-video game system designed to enhance player interaction. Its development was motivated by the various properties of board games: board games are mobile, highly interactive, provide a nonoriented interface, and allow dynamic number of players and house rules. They are also limited to a fairly static environment, don't allow players to save th game state, and have simple scoring rules. On the other hand, computer games provide complex simulations, impartial judging, evolving environments, suspension of disbelief, and the ability to save the game sate. But computer games often support interaction with the system, rather than with other players. 
      The goal of developing the False Prophets hybrid game was to leverage the advantages of both of computer games and board games, encouraging interaction between players. 

      In False Prophets, players use tangible pieces to move around a digital board projected onto a touch-sensitive table. The playing pieces are equipped with a button for simple game operations, while more complex interactions and private information are managed through a hand-held computer. 

      \subsubsection*{Smart Jigsaw Puzzle}
      Is a hybrid tabletop game that augments the physical pieces of a jigsaw puzzle with RFID tags. The underlying RFID reader technology is linked to a PC application that demonstrates a virtual representation of the jigsaw puzzles physical state. 


    \subsection*{4) Location-Aware Games}
    While augmented tabletop games utilize pervasive computing technology to enrich physical game boards, another popular approach in the pervasive gaming field is to regard the whole world, the architecture we live in, as a game board. 

    In {\bf Pirates!} players move around in the physical domain and are presented with location-dependent games on mobile computers they carry with them. Each PDA resembles a pirate ship, and several locations in the physical world are associated with islands the players can visit and experience via games on their PDAs. In contrast to recent pervasive games played outdoor, Pirates is an indoor game with a relatively small playing field. 

    GPS signal reception does not work indoor, and is only moderately accurate outdoors; but it allows us to cover a large playing field. GPS do not support communication, so additional communication channels like wifi or 3G can be used. 

    Some issues with large-scale pervasive games is to deal with the uncertainty in sensing and wireless communication. One solution is to tailor the technical shortcomings and make them a part of the game experience. 

      \subsubsection*{Treasures} 
      Is a pervasive multilayer game played on an outdoor area of several thousand square meters. The game revolves around collecting virtual coins that are hidden in the area. They players need to be at the physical location to collect the coins. A key aspect is the seamless integration of technology into out everyday lives. 
      There will might be areas where there is no wifi connection. This can be integrated into the game play so players can hide in the shadows of the missing connectivity. 

      \subsubsection*{Can you see me now?}
      Combines pervasive gaming action in a real part of a city with online game-play in a virtual model of the game area. In real steets, runners are equippted with GPS and WiFi run around to catch the online players that move through the virtual representation of the streets (real runners and online players). The runners use walkie-talkie to coordinate their movements to catch the online players. The online players sit at a home computer listening to the walkietalkie via stream audio. Text messages can be exchanged for coordination between online players. 

      \subsubsection*{Uncle Roy All Around You}
      Is a successor of ``Can You See Me Now?''. CYSMN managed to mix online and physical games, URAAY picks up the concept of these two domains. But it also integrates more aspects of the real world into the game mechanics. In particular, physical players remains constantly uncertain about which part of the real environment are actual parts of the game. For instance, through communication with online players or the game itself, players in the streets recive hints such as gender or color of other important persons in their vicinity, effectively integrating passers-by into the action. 

      The story is about finding, somewhere in the city, a mysterious character known as Uncle Roy, with the physical players searching in the streets. They are supported by online players who track their progress an aid them with hints about the way to Uncle Roy. 

      \subsubsection*{Catch the flag}
      In a augmented twist on the popular traditional game, smart phones are used as th main interface, enabling physical role play, as players are not confined to desktop computers. Players can move about freely in the real world over wide area outdoor spaces, while maintaining seamless real-time networked contact with other players in both the real and virtual world. 
      This game is an popular outdoor game for children. In the mixed reality version, each team ha at least one knight and one guide, as well as other entities including bombs, traps, flags, magic potions and castles. Throughout the game, guides use a desktop application to assist allied knights by giving guidelines and setting magic potions, which for a short period of time can turn knight into warriors, as well as catch opponents knights by setting traps. Knights navigate and receive guidelines though a smartphone, and can catch the opponents flag by acquiring the Bluetooth object representing the flag in the real world. Any team who can safely bring the opponents flag to its base wins the game. 

    \subsection*{5) Augmented Reality Games}
    Perhaps the most technically advanced pervasive games use augmented reality techniques as a bias. Users see their view augmented with 3D objects registered such that they appear to exist in real space. 

    Augmented reality can be created with different technologies:
    Using head-mounted displays, using images projected on real world surface, or using hand-held devices. 

      \subsubsection*{The characteristics of AR Games}
      In augmented reality, players are not immersed in virtual content, rather, virtual elements are added to the real world. It is stated that the ideal entertainment experiences comes from the combination of physical experience, virtual content, storytelling, and the imagination of the user. Augmented reality offers both physical and virtual aspects, leaving creative designers to simulate the imagination. 

      Many early augmented reality games have focused more on the technology rather then the game design. 

      \subsubsection*{$AR^{2}$ Hockey}
      Players as physical paddles and the puck is virtual. The projects intent was to demonstrate technology, it can serve to illustrate the possibility of new game features. PingPongPlus is also a similar implementation. In it, the ball remains physical, but the system tracks it movement and presents visual effects on the game table using a projector. 

      \subsubsection*{ARQuake}
      The ARQuacke projects demonstrates 

      \subsubsection*{Human Pacman}
      Human players who takes the role of pacman. Pacman and Ghost players can see, via head-mounted displays. The cookies are shown from a first-person shooter perspective. It lacked social interaction, so a helper is introduced.  

      \subsubsection*{Tilt-Pad Pacman}
      A using wearing Head-mounted display sees the 3D world and controls the pacman by tilting a heldhand device. 

      \subsubsection*{Magic Land}
      Is an augmented reality environment in which 3D avatars of live human beings and 3D computer-generated virtual animations play and interact. The environment has two main areas: the recording room and the interactive room. The recording room is where users can have themselves captured and made into live 3D models that will interact in the mixed-reality scene. After being captured, the user can go to the interactive room to play with their own figure. 

      \subsubsection*{AR Worms}

      \subsubsection*{AR Tankwar}

      \subsubsection*{Open Issues in AR Games}
      As well as offering a unique combination of characteristics for game design, augmented reality presents some unique problems. 
      For example, there is limited collaboration , particularly with the use of head mounted displays. 

      The development of interaction metaphors is still immature (paddle, pointer, and lens-based interactions are some examples). While augmented reality may allow players to use their whole bodies to interact with the game, research is needed to determine what metaphors are most effective in different gaming situations. 

      Current augmented reality systems require especially configured hardware and controlled environments. Such systems are expensive to set up, and so commercial game development is not viable. In the near future, hand-held augmented reality games may become popular via mobile phones; but head-mounted displays may require drastic changes in technology or the marketplace to become viable. Fixed installations and other business models may be a option.

      Augmented reality offers many possibilities for game design. Games designed using it may draw the characteristics of both real-world and computer games. However, there are significant challenges due to its expense, difficulties with hardware, and lack of design guidelines. 

