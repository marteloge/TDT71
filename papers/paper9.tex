\chapter*{9) Requirements Engineering and the Creative Process in the Video Game Industry}
  
  \section*{Introduction}

    \begin{itemize}
      \item Requirements like fun and absorbing are not well understood from the perspective of requirements engineering. 
      \item Decrease the cost of delays caused by communication errors. Locate the causes of the most costly errors.
    \end{itemize}

  \section*{Background}

    The ability to precisely communicate and capture stakeholder wants and needs is rare. Members of video game development teams include practioners from such diverse backgrounds as art, music, graphics, human factors, psychology, computer sience, and engineering. 

    \subsubsection*{Emotional Factors}
    Few researchers have investigated emotional factors in requirements engineering. Feks what makes a game fun. Emotial factors is important to make a game success and human factors needs to be investigated.

    \subsubsection*{Language and Ontology}
    The game designer works with the production team to translate the vision to requirements, usually stated in natural language complete with domain spesific terminology. A common language, ontology, or vision is often mentioned as the solution to communication issues between different stakeholders. 

    \subsubsection*{Elicitation, Feedback and Emergence}
    Emergent requirements discovered during the transition from preproduction to production are a significant aspect of the creative process. Communication between preprod and prod.

  \section*{Video Games Development}
  Video games are big business. However, for every advertisement for a newley released game, the trade press reports a large number of projects that fails to reach the market. What is the cuase of these failures? The multidisciplinary nature of video game development process -  with art, sound, gameplay, control systems, human factors interacting with traditional software development creates complecities that may recommend a specialized software engineering methodology for this domain. 

    \subsubsection*{Development Process}
    The game process is divided into two parts: preproduction and production. The first phase is the preproduction resulting in a {\bf Game Design Document (GDD)}. Preproduction is defining the wants and needs before meeting with the development team. The other phase is the production phase. Requirements engineering, with the assistencee of the game designers, transforming the GDD into spesifications. Once the specification is complete, a traditional software development process begins, resulting in the {\bf game artifact}.
    \subsubsection*{The Game Design Document}. Moving from preproduction to production is particularly difficult in video games.

    \subsubsection*{The Game Design Document}
    The game design document is a creative work written by the game designer. If the GDD is stricted to have a formal structure, the creative process may be reduced. The GDD often includes concept statement, tagline, game genre, the story behind the game, chacters, and charcters dialog. It will also include description of how the game is played, the look, the feel, and sound of the game, the levels or missions, the cutscenes, puzzles, animations, special effects, and other elements required. The dangers with Ad hoc approach (no documentation) is the dependence of human memory for capturing decisions and their justifications. Two sets of documents needs to be maintained, and is a big cost in time and the writing stiles of the two documents is very different.  

  \section*{The Transition from Preproduction to Production}
  Requirements errors are some of the most costly to fix. Too many projects violate their preproduction phases and move straight to production. Preproduction is one of the most important phases. Game Design Documents are rarely completed because it needs to be maintained throught the course of production. With time-to-market pressures, it is easy to underestimate the importance of maintainance and it is often given low priority. 

  \section*{Review of Postmortem Columns}
  Game development are a competetive business and there is not much available to read for researchers. Therefore it was used postmortems from the Game Developer Magazine. This include summaries like: explain 5 goals, features of aspects of the project describing what went well and what failed. 

    \begin{enumerate}
      \item {\bf Preproduction:} issues outside the traditional software development process such as inadequate game design or inadequate storyboarding.
      \item{\bf Internal:} issues related to project management and personnel.
      \item {\bf External:} issues outside the control of the development team such as changes in the market or financial conditions.
      \item {\bf Technology:} issues related to the creation or adoption of new technologies. 
      \item {\bf Schedule:} issues related to time estimates and overruns. (a part of a internal issue).
    \end{enumerate}

  Internal factor is a dominating issue in game development. This is often related to the project management issue. ``Inadequate planning'', ``underestimating the scope of tasks'', ``aggressive schedule'', ``clear goals are great when they are realastic'', ``an unrealistic schedule cant be saved without pain''.

  \section*{Examples from real games}
  The resuls from the analysis of the postmortem columns led to the conclusion that weak management of the transition from preproduction to production was a source of many issues in video game development.

    \subsubsection*{Documentation Transformation}
    Creating the requirement specification often requires a priori knowledge of the available technology so that the requirements can be presented in a context. The multiple stakeholder viewpoints, synthesizing a common domain language, numerious nonfunctional requirements, and incomsistencies as the project evolves. The associated costs are significant, leading to a strong management bias toward minimizing the documentation effort.

    \subsubsection*{Implication}
    By its nature as a creative work, a game design document is full of
    implied information. Identify these implications requires careful analysis, understanding the ramifications (a consequence of an action or event) of the implications require significant domain knowledge. 

    \begin{enumerate}
      \item {\bf First level of implications:}
      \item {\bf Second level of implications:}
      \item {\bf Third level of implications:}
    \end{enumerate}

    A question is raised by identifying these three levels of implication: it is more approperiate to follow a traditional iterative process and allow these issues to surface later, or should this feedback be applied as early as possible in the process? Intuitively, eraly feedback is better. However, early feedback could have a negative effect on the creative process of the game designers. 

    \subsubsection*{A Priori Knowledge}
    Domain specific terms, particularly abbrevations and acronyms, are common in working papers.
    One approach is to add professional technical writing resources to the projects. Maintaining an iteration history of sketches, such as the working paper, is challenging. Changes and justifications is an important piece of information later in the developement cycle. These justifications could lead to evolutionary chanegs in the game engine and architecture. 
    In the example, the suggested design was not supported by the technology. The success was achieved through dialog between team members, but the revised requirements and specifications for the final product was never fomally captured. 

    \subsubsection*{Evaluation}
    The multiple design interations due to the technical issues cuased problems with the schedule because of the time used. 

  \section*{Summary and Conclusion}
  Model for video game development that integrates preproduction with production. Project management issues are the greatest contributors to success or failure. In the case of failure, many of these issues can be tracked back to inadequate requirements engineering during the transition from preproduction to production. The introduction of production personnel into the preproduction process may have a negative effect on the creativity of the preproduction team. 


  \section*{Challenges for Requirements Engineering}

  \begin{enumerate}
    \item Communication between stakeholders of different background
    \item Remaining focused on the goal and resisting feature creep
    \item Influence of prior work
    \item Media and technology interaction and integration
    \item The importance of nonfunctional requirements
    \item Gameplay requirements

  \end{enumerate}

    \subsubsection*{Meadia and Technology}
    Creating video game require the creation of numerious software artifacts.

    \subsubsection*{The Importance of NFRs}
    Video games are designed to entertain. Therefore, nonfunctional requirements such as fun, storyline, aesthetics, and flow must dominate their requirements sprecification. However, there are no established practices for capturing and specifying such NFRs. 
    Validation of gaming NFRs is very complex. The link between NFRs and target markets or user demographics has not yet been explored by RE in this domain. 

    \subsubsection*{Gameplay}