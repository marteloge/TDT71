\chapter*{8) From Visual Simulation to Virtual Reality to Games}

    \section*{Defining Serious Games}
    People respond differently to the emotionally charged term game depending on wheter they played or did not play video games while growing up. 

    {\bf Video Game: a mental contest, played with a computer according to certain rules for amuesements, reaction, or winning a stake.}

    Developing a science of games opens a huge potential for the wider application of games in governmental or corporate training, education, health, public policy, and strategic  communication objectives. 

    Electronic Arts defines video games as: ``story, art, and software''. 
    Serious games have more than just a story, art and software. They involve pedagogy: activities that educate or instruct. This addition makes games serious. Pedagogy must be subordinate to story - the entertainment component comes first. 

    \section*{Creating a Science og Games}
    The development and wide release of the Americas Army game began an revolution in thinking about the potential role of video games for nonentertainment domains. A study of the game was conducted to see if  it could be used for training. 
    Example of use: when new recruits had trouble with the rifle or the obstacles course, his team had those recruits play ``Americas Army'' and required them to complete those levels in the game. When the reqruits did so, they then went back to the range and usually passed the range tests. This made us wonder what other applications would benefit from game-based instruction.
    Mothers complained about their sons playing the game all the time and their kids knew everything about the army. These comments led the researchers to wonder how much of science and math education could be taught via games (learning from other than formal teaching). This is called ``first-person education''. 

    \section*{Game Production Challenges}
    First-person shooter games are recently the only games that have reuseable engines. Game engines are costly to develop! The demand for better computer characters and story increases with the complexity of visual displays and more complex games. Game play innovation is becomming a competitive necessity. The big hits have been sports, first-person shooters, adventure, and sims-type games. For the game industry to continue to grow, additional genres must become more sophisticated with better backstories and better game-play technologies. 

    \section*{Games Reseach Agenda}
    To influence the future of both serious and entertainment games, developers must create a research agenda that transforms the game production process form a handcrafted, labor-intensive effort to one shorter, more predictable production timelines that still manages to provide innovations and increased complexity. This reaseach agenda have three components: infrastructure, cognitive game design, and immersion.

      \subsubsection*{Infrastructure}
      Underlying software and hardware necessary for developing interactive games include massiveley multiplayer online game architectures, game engines and tools, streaming media, next-greneration consoles, and wireless and mobile devices. Game engines and tools will be vital to researchers bent on attacking the lack-of-reuse problem in gaming. This can also move games to the government and corporate sectors. Developers need an open source game engine that includes a development toolset as wideley available and utilized as Linux. Streaming media will play prominent role in the delivery of dynamic content to PC -based games and mobile devices. 

      \subsubsection*{Cognitive game design}
      Taking a cognitive approach to game developemt will give developers the tools to create theories and methods for:

        \begin{itemize}
          \item modeling and simulating computer characters, story, and human emotion
          \item analyzing large-scale games
          \item innovating new game genres and play styles
          \item integrating pedagogy with story  in the interactive game medium 
        \end{itemize}

      The Sims for a serious purpose, such as training aid for nursing. Developers can use this approach to model and simulate hospital operations in game form, providing an immersive experience for the nurse trainee. 

      It must be determined what happens during game play and how the experiences affects the players. The game industry has already witnessed tha failure of edutainment, an awkward combination of educational software sprinkled with game-like interfaces and cut dialog. This shows that story must come firstand that research must focus on combining instruction with story creation and the game development process. 

      \subsubsection*{Immersion}
      Creating technologies that engage the game players mind via:
        \begin{itemize}
          \item computer graphics, sound, and haptics
          \item affective computing - sensing human state and emotion
          \item advanced user interfaces
        \end{itemize}
      In the next years, low cost sensors will become available that measures the players emotional state and provides this information as input when running the game. Game designers must understand these things if they are to engineer and implement these capabilities so that these features behave predictabily and reabily.
      Advanced user interfaces will become key as computing moves from the standard desktop PC to mobile platforms. Much can be gained by studying how the game industry has developed almost universal interfaces that let gamers transision seemlessly from one game to another. To make significant progress in deploying serious games, developers must understand interfaces from the game perspective. 

    \section*{Serious Games}
    Applying games and simulation technology to nonentertainment domains results in serious games. 

    \section*{Gamepipe Laboratory}
    The first serious games must be constructed in carefully controlled university environments. The pipeline laboratorys mission is to research and develop interactive game technologies that dramatically shorten the production timeline, provide supporting technologies for increasing the complexity and innovation in produced games, and enable the creation of serious and entertainment games for government and corporate sponsors. The GamePipe Laboratory performs research and development required for the next generation game production. It produces serious games and game prototypes as sponsored projects. It also links with, and develops educational programs in game developement and game-related research and build cross-campus interdisiplinary teams to solve next-generation game technology and development problems. 
