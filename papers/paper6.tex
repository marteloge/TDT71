\chapter*{6) GameFlow: A Model for Evaluating Player Enjoyment in Games}

  \section*{Enjoymen in Media and Games}
  Player enjoyment is the single most important goal for computer games. If Players do not enjoy the game, they will not play the game. Research is often focusing on usability (interface, mechanics and gameplay), rather than enjoyment in games. Many different models have been developed to explain and analyze media enjoyment:

    \begin{itemize}
      \item {\bf Attitude} is a psychological tendency expressed by evaluating a particular entity with some degree of favor or difavor.
      \item {\bf Parasocial interaction} takes place when an audience member develops a parasocial relationship with a media character by talking to the character, imagining, or discussing the life of the character. 
      \item {\bf Transportation theory} suggests that the experience of enjoyment is heighted by an immersion in a narrative world, as well as from the consequence of that immersion. Transportation is a melding of attention, imagery, and feelings.
      \item {\bf The theory of disposition} reletes positive or negative attitudes toward media characters to moral evaluations of their actions, as well as empathy for the main character. 
      \item {\bf Cognition} in relation to media enjoyment involves viewers making judgements on, for instance, the characters ethics, interest, and intelligence. 
    \end{itemize}

  Each theories aims to analyse and understand the enjoyment in terms of one specific aspect or concept. Idividually these theories are fairly narrow. 
  The aim of this study is to develop and validate a model of player enjoyment in games that is based on flow. 

  \section*{Enjoyment and Flow}
  A researcher did analyse people who spend large amount of time and effort on activities that are difficult, but provide no external rewards (money, status, etc), such as chess players and rock climbers. Later he analysed ordinary people by asking them how it felt when their lifes were at their fullest and when what they did was most enjoyable. He found that optimal experience, or flow, is the same the world over. 

  {\it ``Flow is an experience so gratifying that people are willing to do it for its own sake, with little concern for what they will get out of it, even when it is difficult or dangerous''}. 

  Flow experience consists of 8 elements: 
  \begin{enumerate}
    \item a task can be completed;
    \item the ability to concentrate on the task;
    \item that concentration i possible because of the task have clear goals;
    \item that concentration i possible because of the task provides feedback;
    \item the ability to exercise a sense of control over actions;
    \item a deep but effortless involement that removes awareness of the frustration of everyday life;
    \item concern of self disappears, but sense of self emerges stronger afterwards;
    \item the sense of the duration of time is altered
  \end{enumerate}

  \section*{Adapting Flow to Games}

    \subsubsection*{Concentration}
    To be enjoyable, a game has to require concentration, and the player must be able to concentrate on the game. The more concentration a task requires in terms of attention and workload, the more absorbing it will be. Games should grab the players attention quickly and maintain it throughout the game. It is important to increase the players workload but maintain an approperiate level for the players perceptual, cognitive, and memory limits. Also, the players shouldnt be burdened with tasks that dont feel important. 

    \subsubsection*{Challenge}
    Challenge is consistently identified as the most important aspect of good game design. Games shouldnt be sufficiently challenging, match the players skill level, vary the level of difficulty, and keep approperiate pace (Not discouraginly hard or boringly easy). Satisfaction in games comes from accomplishing difficult tasks, challenging and surpassing opponents, testing skills, mastering skills, reaching a desired goal, and coping with danger.  (the feeling of triumph).

    \subsubsection*{Player Skills}
    For games to be enjoyable, they must support player skill development and mastery. It is necessary that player develops their skills at playing the game. Intersting tutorials. Rewards are an important part of learning to play a game (learning rights and worngs). Players should also have enough information to start playing the game. 

    \subsubsection*{Control}
    In order to experience flow, players must be allowed to exercise a sense of control over their actions. Players should be able to translate their intentions into in-game behavior and feel control of the actual movements of their character. 
    The game should be easy to use, allowing players to start the kind of game they want, turn the game on and off, and save the game in different states. 
    Errors in the game can make players feel they have lost control, especially if the errors or consequence are out of the players control. 
    The changes that players make in the world should be persistent, and be noticeable when players backtrack to where they were before to show that they had impact on the world. 
    The should not be any single optimal strategy for winning and gameplays should be balanced with multiple paths throught the game or ways to win. 

    \subsubsection*{Clear Goals}
    Games should present the player with a clear overriding goal early in the game. Each level should have multiple goals. 

    \subsubsection*{Feedback}
    Players must recieve feeback at approperiate times. Feedback on direction, distance, progress towards goal. In-game interfaces and sound can deliver feedback. Games should also provide immediate feedback for players actions. 

    \subsubsection*{Immersion}
    Players should experience deep but effortless involment in a game. Immersion, engagement, and absoption are concepts that are highlighted. Many game-players devoting entire nights or weekends playing games without being concurrently aware of doing so. Players often have a high level of emotional investment in games due to the time, effort, and attention put into playing. People play games to think thoughts and feel emotions that are related to work, to calm down after a hard day or to escape from everyday worries. Games are often seen as a form of escape from the real world or social norms, or as a way to do things that people otherwise lack the skills, resources, or social permission to. 

    \subsubsection*{Social Interaction}
    Games should support and create oppertunities for social interaction. Social interaction is not an element of flow. However, it is clearly a strong element of enjoyment in games, as people play games for social interaction, whether or not they like games or the game they are player. To support social interaction, games should create oppertunities for player competition, cooperation, and connection. 


  \section*{Validating the Gameflow Critera}


    \subsubsection*{Evaluation: Warcraft 3}

      \begin{itemize}
        \item {\bf Concentration:} multitude of high-quality stimuli, quickly grabbing the players attention and is held throughout the game, there is not a lot of unimportant tasks, the mission and goal feels important, and the workload is high.
        \item {\bf Challenge:} Not met in skirmish mode since the opponent AI is hard, the difficulty is increasing, can switch race with new challenges. 
        \item {\bf Player Skills:} Players are able to start playing Warcraft 3 immediatley, the tutorial is otional is you are an expert, new objects and features are introduced one at a time, Players are rewarded for their skill development with items, experience, and skills for their heroes. 
        \item {\bf Control:} allows players to feel in control of every aspect of the game, there is no problem with errors or recovering from errors since the game is very polished, the players have freedom and control of their actions and strategies that they use, Players can use their own strategies and choose a suited race for their strategy. 
        \item {\bf Clear Goals:} present both overriding and intermediate goals
        \item {\bf Feedback:} Players are immediatley notified when goals or a mission objectives are completed and can check the status of their goals, subgoals, and completed goals. There is a notification when something new is happening. Players recieve their scores, broken down into heroes, units, and resources. 
        \item {\bf Immersion:} It is easy to become engrossed in Warcraft 3, as there is so much to concentrate on, many tasks to perform, things to monitor, as well as graphics, sound and animation. 
        \item {\bf Social Interaction:} supports competition and cooperation between players, as well as social communities inside and outside the game. 
      \end{itemize}

    \subsubsection*{Evaluation: Lords of EverQuest}

       \begin{itemize}
        \item {\bf Concentration:} numerous stimuli which vary in quality. attractive game world, visually appealing characters, only two or three responses that becomes repetitive. No background story, no motivation or variation. The player is required to focus on unimportant tasks, and the workload is too low. 
        \item {\bf Challenge:} The challenge is below average. Challenge is poor for experienced players, Missions are straightforward, Little variation between missions, Units are very unbalanced. 
        \item {\bf Player Skills:} ok tutorials, familiar but poor interface, Icons are hard to understand because of the details, 
        \item {\bf Control:} do not have full control of several aspects of the game, including units, interface, and gameplay. One good attribute is that the interface is customizeable. If a player fails a mission, the player doesnt find out until the end and need to start over and replay the whole mission. The player does not have freedom in the strategies. Consequently, there is no verety in play styles or strategies available to the player. 
        \item {\bf Clear Goals:} do not present any overriding goal beside that there is a war, no goals are ever really presented. Players are told what to do, but not why. 
        \item {\bf Feedback:} provieds sufficient feedback in general. The player is not given feedback on a mission until the end. Statistics when they level up are not visible. There is no difference in the terrain or visualisation of available places to build. 
        \item {\bf Immersion:} the game is too slow for players to become immersed as there is not enough to concentrate on and not enough challenge. Too much time spent waiting. There is not enough background, character development or storyline for the players to get emotionally involved in the game or connected to characters. 
        \item {\bf Social Interaction:} the game supports competition and cooperation between players. 
      \end{itemize}

    \subsubsection*{Comparisation}


  \section*{Discussion and Conclusions}